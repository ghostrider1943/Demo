\documentclass{article}
\title{\textbf{Planck’s Equation}}
\date{June 2022}
\author{Parth borkar NA21B052}

\usepackage{graphicx}


\begin{document}
  \maketitle
\large
\paragraph{} \textbf{Plank's law} describes the spectral density of electromagnetic radiation emitted by a black body in thermal equilibrium at a given temperature T, when there is no net flow of matter or energy between the body and its environment.

\vspace{1cm}


\boldmath
\begin{equation}
  E=h\nu
\end{equation}

\vspace{1cm}

This formula is considered one of the most important physics formulas, as it is responsible for the birth of quantum mechanics, also television and solar cells.  Max Planck postulated in 1900, that energy was quantised and could be emitted or absorbed only in integral multiples of a small unit, which he called “energy quantum”.
\vspace{1cm}

\begin{tabular}{|c|l|}
    \hline
    $E$ & Energy of a photon\\
    $h$ & Plank's constant (6.6262x10^{-34}J.s)\\
    $\nu$ & Frequency of photon\\
    \hline
\end{tabular}


\end{document}
